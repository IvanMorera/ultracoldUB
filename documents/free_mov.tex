\section{Movimiento libre}

\subsection{Introducci\'on}
Para empezar a adentrarnos en el mundo de la Mec\'anica Cu\'antica, estudiaremos posiblemente el caso m\'as sencillo existente. \'Este es el estudio del movimiento de una part\'icula en reposo. Intuitivamente seg\'un la f\'isica cl\'asica la part\'icula permanecer\'a en reposo en su posici\'on, l\'ogicamente. Estudiemos este movimiento des del punto de vista de la Cu\'antica, donde posiblemente quedemos fascinados por su interpretaci\'on.
\\
\subsection{Simulaci\'on}
Para empezar con la simulaci\'on entramos en el m\'odulo $Wave Packet Dispersion$. Una vez dentro pulsamos el bot\'on $Demo 1$, autom\'aticamente se iniciar\'a el c\'alculo de una simulaci\'on con una configuraci\'on adecuada. Una vez finalizado el tiempo de carga, debemos ver una imagen como la mostrada en Fig.~\ref{Fig:free_mov} .Podemos observar en la gr\'afica central la evoluci\'on de nuestra part\'icula. Para controlar la simulaci\'on existen tres botones: $Go$ $on$, $Go$ $back$, $Pause$ que evolucionan el sistema hacia adelante, atr\'as o lo pausan, respectivamente. 
\\

\begin{figure}[tb]
	\centering
	\includegraphics[width=0.9\linewidth]{free_mov.png}
	\caption{Ejemplo de la interfaz para el caso del movimiento libre}
	\label{Fig:free_mov}
\end{figure}

La figura coloreada de azul muestra la probabilidad de encontrar nuestra part\'icula. La figura coloreada de amarillo muestra la probabilidad de encontrar nuestra part\'icula con una cierta velocidad.
\begin{itemize}
	\item \textbf{¿Te atreves?} (Nivel bajo): D\'onde es m\'as probable que encontremos nuestra part\'icula al inicio del movimiento? Y al final? Lo mismo para la velocidad.
\end{itemize}


\begin{itemize}
	\item \textbf{¿Te atreves?} (Nivel medio): A pesar de que nuestra probabilidad vaya cambiando con el tiempo, d\'onde estar\'a situada la media respecto la posici\'on y velocidad de nuestra part\'icula?
	
	Pista: Entrar en la secci\'on $Mean$ $Value$ $X$.
\end{itemize}


As\'i podemos ver que a pesar de que tengamos toda una zona de probabilidad donde encontrar nuestra part\'icula, en media tenemos que estar\'a en el centro. Justamente lo esperado por la f\'isica cl\'asica, la part\'icula se queda en reposo en un punto determinado. En la secci\'on $Mean$ $Value$ $X$ podemos encontrar tambi\'en la evoluci\'on de los valores de la desviaci\'on t\'ipica de estas distribuciones (anchura de ellas).
\\

\begin{itemize}
	\item \textbf{¿Te atreves?} (Nivel bajo): C\'omo avanza la anchura de las dos distribuciones respecto el tiempo?
\end{itemize}


Ahora veremos una analog\'ia con el movimiento visto anteriormente y la dispersi\'on de la luz al travesar un medio transparente en el cual se refracta. Pulsamos el bot\'on $Dispersion$.
\\

Aqu\'i podemos ver una simulaci\'on que ilustra el efecto de la dispersi\'on de la luz. Incidimos luz blanca en un medio; como la luz blanca est\'a compuesta por diferentes colores (frecuencias), cada componente de la luz obtendr\'a una velocidad diferente, esto producir\'a que la luz blanca se descomponga en todos los colores.
\\

Con esto hemos visto una clara analog\'ia entre la dispersi\'on de la luz y la dispersi\'on de la probabilidad de encontrar una part\'icula libre. Esto es debido a que, como en la luz blanca, tenemos luz formada por diferentes frecuencias, en nuestra distribuci\'on pasa algo muy similar. Cada componente que forma nuestra distribuci\'on posee una velocidad diferente, esto hace que se disperse con el tiempo, al igual que la luz. As\'i vemos que al inicio tenemos un r\'egimen, m\'as o menos cl\'asico, las probabilidades de encontrar la part\'icula en posici\'on o velocidad $0$ son realmente grandes, pero al avanzar en el tiempo estas probabilidades se dispersan y perdemos nuestro r\'egimen cl\'asico. A pesar de que pensemos que nuestra part\'icula est\'a en reposo, podemos encontrarla en diferentes puntos de nuestro sistema, asombroso!
