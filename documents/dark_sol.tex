\section{Solitones oscuros y condensados de Bose-Einstein}

\subsection{Introducci\'on}
Sigamos con la Mec\'anica Cu\'antica, pero ahora daremos un gran salto hacia delante. A diferencia de lo visto anteriormente donde est\'abamos estudiando el comportamiento de una \'unica part\'icula, ahora nos centraremos en estudiar el comportamiento de una gran agrupaci\'on de part\'iculas. Entramos en el mundo de los gases ultrafr\'ios.
\

Imaginemos que tenemos un gas y lo enfriamos hasta tal punto de llegar al 0 absoluto (0 grados Kelvin). En estas condiciones nuestro gas empezar\'a a presentar efectos cu\'anticos, entramos en un nuevo estado de la materia; m\'as all\'a de los gases, l\'iquidos y s\'olidos. Veamos este asombroso efecto.
\\

Entramos en el m\'odulo $Dark Soliton Simulation$. Nada m\'as entrar podemos ver el gas del cu\'al est\'abamos hablando (parte blanca). Para que el gas no se escape lo hemos confinado (atrapado) con un potencial arm\'onico, por eso adopta tal forma. En esta secci\'on nos centraremos en los llamados 'dark solitons', los cuales los podemos entender como 'agujeros' en nuestro gas. Es decir, zonas donde la densidad del gas disminuye dr\'asticamente.


\subsection{Simulaci\'on}
Para empezar pulsamos el bot\'on $Demo 1$ y esperamos a que finalice el c\'alculo. En la ventana central podemos ver nuestro gas pero observamos como ahora se ha formado un dark soliton (agujero) en la parte derecha de \'este. Si ahora pulsamos $Go On$ y as\'i la simulaci\'on se pone en marcha, podemos ver como el agujero empieza a oscilar sobre un punto central. Sorprendente!

\begin{itemize}
	\item \textbf{¿Te atreves?} (Nivel medio): Sabr\'ias explicar el motivo por el cual el agujero empieza a oscilar?
	
	Pista: recuerda las condiciones a las cuales est\'a sometido el gas.
\end{itemize}

Lo que est\'a pasando es ciertamente incre\'ible. Nuestro agujero (ausencia de part\'iculas en una zona), se est\'a comportando como si fuera una part\'icula. 

\begin{itemize}
	\item \textbf{¿Te atreves?} (Nivel bajo): Si ahora entramos en la secci\'on $DENSITY MAP$ podemos ver como avanza nuestro 'dark soliton' con el tiempo. Te resulta familiar esta forma? Podr\'ias compararla con alguno de los casos vistos anteriormente?
\end{itemize}

Finalmente volveremos a entablar una analog\'ia con la f\'isica cl\'asica y, volveremos a recurrir al movimiento del muelle. Para ello presionamos el bot\'on $Guess Function$.

\begin{itemize}
	\item \textbf{¿Te atreves?} (Nivel bajo): Dados los par\'ametros $A$ y $w$ (sabr\'ias explicar a que corresponden?) puedes ajustarlos de tal forma que nuestro muelle coincida con el movimiento de el 'dark soliton'?
\end{itemize}

\begin{itemize}
	\item \textbf{¿Te atreves?} (Nivel muy alto): Una vez encontrados los par\'ametros adecuados para que coincidan los movimientos, si como dada sabemos que la frecuencia del movimiento arm\'onico es equivalente a la unidad. Podr\'ias explicar por qu\'e el 'dark soliton' no sigue esta frecuencia?
\end{itemize}

Vamos a concluir con un repaso de lo que hemos visto, ya que han aparecido muchos conceptos. Tenemos un conjunto de part\'iculas (gas), el cual se ha enfriado hasta el 0 absoluto. Una vez llegados a este punto, se ha atrapado el gas con un potencial arm\'onico. Ahora que tenemos nuestro estado atrapado le hemos hecho un 'agujero' (dark soliton), el cual se ha empezado a mover como una part\'icula! \'Esto \'ultimo es realmente asombroso. 
\\

Tambi\'en hace falta denotar un peque\~{n}o detalle que puede que se le haya escapado al usuario y es algo crucial en el comportamiento de los solitones; y es que, el solit\'on al empezar a moverse no cambia su forma (en nuestro caso anchura). Par\'emonos a pensar en este hecho un segundo. Imaginemos que tenemos un gas cl\'asico y, a \'este se le practica un agujero, inmediatamente el gas tender\'a en la medida de lo posible a volver a llenar ese espacio vac\'io, totalmente al contrario de lo que pasa en nuestro ejemplo.
\\

Con esto finaliza esta secci\'on, pero no nos quedamos aqu\'i. A continuaci\'on se presentar\'a un ejemplo de hasta que punto estos 'agujeros' (dark soliton) se comportan como part\'iculas, vamos a ver un caso extremo realmente fascinante, el llamado p\'endulo de Newton cu\'antico.